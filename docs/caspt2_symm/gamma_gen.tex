\documentclass[12pt]{article}
\usepackage[utf8x]{inputenc}
\usepackage[english]{babel}
\usepackage[T1]{fontenc}
\usepackage{color}
\usepackage{amsmath}
\usepackage{amssymb}
\usepackage{textcomp}
\usepackage{array}
\usepackage{booktabs}
\usepackage[font=small,format=plain,labelfont=bf,up,textfont=it,up]{caption}
\usepackage{longtable}
\usepackage{calc}
\usepackage{setspace}
\usepackage{hhline}
\usepackage{ifthen}
\usepackage{lscape}
\usepackage{pdfpages}
\usepackage{geometry}
 \geometry{
 a4paper,
 total={170mm,257mm},
 left=20mm,
 top=20mm,
 bottom=20mm,
 right=20mm,
 }
\usepackage{cite}
\linespread{1.3}

\begin{document}
\begin{equation*}
\sum_{\substack{ x_{1}x_{2}...\\ y_{1}y_{2}... \\ ...}} X_{x_{1}x_{2}...} Y_{y_{1}y_{2}...} ...
\sum_{I}\sum_{J}
\langle I | \hat{E}^{\dagger}_{\Omega} a^{\dagger}_{x_{1}} a_{x_{2}}...a^{\dagger}_{y_{1}}a_{y_{2}}....| J \rangle 
 c^{M \dagger}_{I}c^{N}_{J}
\end{equation*}
\begin{equation}
=
\sum_{\substack{ x_{1}x_{2}...\\ y_{1}y_{2}... \\ ...}} X_{x_{1}x_{2}...} Y_{y_{1}y_{2}...} ...
\sum_{I}\sum_{J}
\langle I | a_{\omega_{1}} a_{\omega_{2}}.. .a^{\dagger}_{x_{1}} a_{x_{2}}...a^{\dagger}_{y_{1}}a_{y_{2}}....| J \rangle 
 c^{M \dagger}_{I}c^{N}_{J}
\label{eqn:basic_2nd_quantized_projector}
\end{equation}
 \noindent where the $\hat{E}^{\dagger}_{\Omega}$ is an operator
which excites electrons from orbitals in the space used to represent
the reference wavefunctions, $\{|N\rangle\}$, and into some
virtual space.  \\

\noindent An illustrative example is provided by the XMS-CASPT2 amplitude equation: 
\begin{equation*}
\sum_{\Omega'}
\langle \tilde{M} | \hat{E}^{\dagger}_{\Omega} (\hat{H} - \epsilon_{L}) \hat{T}^{LM}_{\Omega'}| \tilde{N} \rangle 
+
\langle \tilde{M} | \hat{E}^{\dagger}_{\Omega}\hat{H}| \tilde{L} \rangle 
\end{equation*}

For now we can just consider the second the second term, which written out with all summations over orbitals and 
ci-indexes made explicit is
\begin{equation}
X_{\omega_{1}\omega_{2}\omega_{3}\omega_{4}}
= 
\sum_{ H_{1}H_{2}H_{3}H_{4}} 
\sum_{I}\sum_{J}
\langle I |
\hat{a}_{\omega_{4}} \hat{a}_{\omega_{3}}\hat{a}^{\dagger}_{\omega_{2}}\hat{a}^{\dagger}_{\omega_{1}}
\hat{a}^{\dagger}_{h_{1}} \hat{a}^{\dagger}_{h_{2}}\hat{a}_{h_{3}}\hat{a}_{h_{4}}| J \rangle  
c^{M \dagger}_{I}c^{N}_{J} H_{h_{1}h_{2}h_{3}h_{4}}
\label{eqn:XH_example}
\end{equation}
This cannot be solved using the same techniques as where all the orbital indexes are summed over; rearranging all the terms into 
normal order would  cause contractions with the $\omega$ indexes of $X$. Still, we want a method of evaluation which only
involves contraction of the operator tensors with a reduced density matrix (or similar) with only active indexes.

\noindent To accomplish this, imagine splitting the term up into 
two subsections using the resolution of identity\footnote{This is to illustrate the rationale behind
the algorithm; such a resolution of identity is not required in the final proceedure.}.

\begin{equation}
X_{\omega_{1}\omega_{2}\omega_{3}\omega_{4}}
= 
\sum_{ h_{1}h_{2}h_{3}h_{4}} 
\sum_{IJK}
\langle I |
\hat{a}_{\omega_{4}} \hat{a}_{\omega_{3}}\hat{a}^{\dagger}_{\omega_{2}}\hat{a}^{\dagger}_{\omega_{1}}
| K \rangle \langle K | 
\hat{a}^{\dagger}_{h_{1}} \hat{a}^{\dagger}_{h_{2}}\hat{a}_{h_{3}}\hat{a}_{h_{4}}| J \rangle  
c^{M \dagger}_{I}c^{N}_{J} h_{h_{1}h_{2}h_{3}h_{4}}
\label{eqn:X_RI_H_example}
\end{equation}
\noindent We can now rearrange this so that the second BraKet $\langle K | ... | J \rangle$ is a sum of \emph{anti}-normal ordered terms, e.g.,
\begin{equation}
\sum_{ h_{1}h_{2}h_{3}h_{4}} 
\sum_{IKJ}
\langle I |
\hat{a}_{\omega_{4}} \hat{a}_{\omega_{3}}\hat{a}^{\dagger}_{\omega_{2}}\hat{a}^{\dagger}_{\omega_{1}}
| K \rangle \langle K | 
\hat{a}_{h_{3}}\hat{a}_{h_{4}}\hat{a}^{\dagger}_{h_{1}} \hat{a}^{\dagger}_{h_{2}}| J \rangle  
c^{M \dagger}_{I}c^{N}_{J} h_{h_{1}h_{2}h_{3}h_{4}}
\label{eqn:X_RI_H_anti_normal1}
\end{equation}
It is apparent that 
\begin{equation}
|K\rangle = \hat{a}_{\omega_{1}}^{\dagger} \hat{a}_{\omega_{2}}^{\dagger}\hat{a}_{\omega_{3}}\hat{a}_{\omega_{4}} |I \rangle
\label{eqn:KI}
\end{equation}
\begin{equation}
|K\rangle =\hat{a}_{h_{3}}\hat{a}_{h_{4}}\hat{a}^{\dagger}_{h_{1}} \hat{a}^{\dagger}_{h_{2}}|J \rangle
\label{eqn:KJ}
\end{equation}
\noindent Now we split the $\mathbf{X}$ into three kinds of blocks. Blocks of the first kind are such that no creation index is in the same 
range as any annihilation index. In blocks of the second kind, one pair of creation and annihiliation indexes may exist in the same space.
In blocks of the final kind, all indexes are in active indexes\footnote{Typically, the CASPT2 amplitudes have no such block.}. \\
 
\noindent If we consider blocks where all indexes are different,
and in the special case where $I\rangle$ and $|K\rangle$ are defined within the same space,
then it is apparent from (\ref{eqn:KI}) and  (\ref{eqn:KJ}) that every non-active creation $\omega$ index must
cancel with a non-active annihilation $h$ index. Suppose we are dealing with the block of $\mathbf{X}$ such that \\
$\omega_{1} = $ core,\\  
$\omega_{2} = $ core,   \\
$\omega_{3} = $ virtual, \\
$\omega_{4} = $ virtual.\\ 
\noindent  This leads to contributions to the block $X_{\omega_{1}\omega_{2}\omega_{3}\omega_{4}}$ 
\begin{equation*}
\sum_{h_{1}h_{2}h_{3}h_{4}} 
\sum_{IJK}
\langle I |
\hat{a}_{\omega_{4}} \hat{a}_{\omega_{3}}\hat{a}^{\dagger}_{\omega_{2}}\hat{a}^{\dagger}_{\omega_{1}}
\hat{a}_{h_{3}}\hat{a}_{h_{4}}\hat{a}^{\dagger}_{h_{1}} \hat{a}^{\dagger}_{h_{2}}| J \rangle  
c^{M \dagger}_{I}c^{N}_{J} h_{h_{1}h_{2}h_{3}h_{4}}( \delta_{h_{3},\omega_{1}}\delta_{h_{4},\omega_{2}} +  \delta_{h_{3},\omega_{2}}\delta_{h_{4},\omega_{1}} ).
\end{equation*}
\begin{equation}
=\sum_{h_{1}h_{2}h_{3}h_{4}} 
\sum_{IJK}
\langle I |
\hat{a}_{\omega_{4}} \hat{a}_{\omega_{3}}\hat{a}^{\dagger}_{h_{1}} \hat{a}^{\dagger}_{h_{2}}| J \rangle  
c^{M \dagger}_{I}c^{N}_{J} h_{h_{1}h_{2}h_{3}h_{4}}( \delta_{h_{3},\omega_{1}}\delta_{h_{4},\omega_{2}} +  \delta_{h_{3},\omega_{2}}\delta_{h_{4},\omega_{1}} ).
\label{eqn:XRIHmatch}
\end{equation}
\noindent We can now reorder the $h$ indexes into normal-ordering, and repeat the process, only this time pairing every non-active annihilation
$\omega$ index with a non-active creation $h$ index of the same range:
\begin{equation*}
\sum_{h_{1}h_{2}h_{3}h_{4}} 
\sum_{IJK}
\langle I |
\hat{a}_{\omega_{4}} \hat{a}_{\omega_{3}}\hat{a}^{\dagger}_{h_{1}} \hat{a}^{\dagger}_{h_{2}}| J \rangle  
c^{M \dagger}_{I}c^{N}_{J} h_{h_{1}h_{2}h_{3}h_{4}}
( \delta_{h_{3},\omega_{1}}\delta_{h_{4},\omega_{2}} +  \delta_{h_{3},\omega_{2}}\delta_{h_{4},\omega_{1}} )
( \delta_{h_{1},\omega_{3}}\delta_{h_{2},\omega_{4}} +  \delta_{h_{1},\omega_{3}}\delta_{h_{2},\omega_{3}} ).
\end{equation*}
\begin{equation}
\sum_{h_{1}h_{2}h_{3}h_{4}} 
\sum_{IJK}
\langle I|J\rangle c^{M \dagger}_{I}
c^{N}_{J} h_{h_{1}h_{2}h_{3}h_{4}}
( \delta_{h_{3},\omega_{1}}\delta_{h_{4},\omega_{2}} +  \delta_{h_{3},\omega_{2}}\delta_{h_{4},\omega_{1}} )
( \delta_{h_{1},\omega_{3}}\delta_{h_{2},\omega_{4}} +  \delta_{h_{1},\omega_{3}}\delta_{h_{2},\omega_{3}} ).
\label{eqn:XRIHmatch2}
\end{equation}
This process is then repeated for all the different blocks  of $\mathbf{X}$. As is apparent from the above the dimension of the 
density matrix can be reduced substantially, but this is at the cost of extensive reordering of the tensor $\mathbf{h}$. However, 
provided symmetry of $\mathbf{h}$ is properly taken into account many of these reorderings are equivalent or closely related,
enabling many of the terms in the reordering to be merged. It should be noted that $\langle I | J \rangle $ may never equal $1$ 
in the relativistic case, as the $ | I\rangle $ and $| J \rangle $ may be from different spin sectors.\\ 

\noindent This method can be applied to more complicated terms, e.g.,
\begin{equation}
\sum_{ \substack{t_{1}t_{2}t_{3}t_{4}\\ f_{1}f_{2}}} 
\sum_{IKJ}
\langle I |
\hat{a}_{\omega_{4}} \hat{a}_{\omega_{3}}\hat{a}^{\dagger}_{\omega_{2}}\hat{a}^{\dagger}_{\omega_{1}}
| K \rangle \langle K | 
\hat{a}_{f_{2}}\hat{a}_{t_{3}}\hat{a}_{t_{4}}\hat{a}^{\dagger}_{f_{1}}\hat{a}^{\dagger}_{t_{1}} \hat{a}^{\dagger}_{t_{2}}| J \rangle  
c^{M \dagger}_{I}c^{N}_{J} T_{t_{1}t_{2}t_{3}t_{4}}f_{f_{1}f_{2}}
\label{eqn:X_RI_fT_anti_normal1}
\end{equation}
Which for the block $X_{\omega_{1}\omega_{2}\omega_{3}\omega_{4}}$ with 
$\omega_{1}$ and $\omega_{2}$ virtual, and $\omega_{3}$ and $\omega_{4}$ active would yield 

\begin{equation}
\sum_{ \substack{t_{1}t_{2}t_{3}t_{4}\\ f_{1}f_{2}}} 
\sum_{IKJ}
\langle I | \hat{a}_{w}\hat{a}_{x}| J \rangle  
c^{M \dagger}_{I}c^{N}_{J} T_{t_{1}t_{2}t_{3}t_{4}}f_{f_{1}f_{2}}
( \delta_{t_{1},\omega_{1}}\delta_{t_{2},\omega_{2}} +  \delta_{t_{1},\omega_{2}}\delta_{t_{2},\omega_{1}} + ... ),
\label{eqn:XfTend}
\end{equation}
\noindent where the $w$ are $x$  are the non-contracted active indexes.
Note that as all the $\omega$ destructors are active, none of the $f$ or $t$ indexes are contracted with them,
resulting in only one bracket. More importantly, there is a reduced density matrix, however as all the indexes on
$X$ are now paired up. This results in a contraction between a two index reduced density matrix, and a six index
tensor, formed from $\mathbf{T}\otimes\mathbf{f}$, with a sum over all the appropriate reordering relations 
specified by the $\delta$ functions. \\ 

\noindent As a rule, if the projector has $n$ different indexes, and the total number of indexes on the operator is $m$, e.g.,
$m=6$ for $T_{t_{1}t_{2}t_{3}t_{4}}f_{f_{1}f_{2}}$, then we will need to contract several different orderings of an $m$ rank
tensor with and $m-n$ rank reduced density matrix. 
\end{document}
